\documentclass[10pt,landscape]{article}
\usepackage{multicol}
\usepackage{calc}
\usepackage{ifthen}
\usepackage[landscape]{geometry}
\usepackage{amsmath,amsthm,amsfonts,amssymb}
\usepackage{color,graphicx,overpic}
\usepackage{hyperref}

% This file was found in the answer at this page: http://tex.stackexchange.com/questions/8827/preparing-cheat-sheets

\pdfinfo{
  /Title (example.pdf)
  /Creator (TeX)
  /Producer (pdfTeX 1.40.0)
  /Author (Lucas)
  /Subject (CS530 Cheatsheet 1)
  /Keywords (pdflatex, latex,pdftex,tex)}

% This sets page margins to .5 inch if using letter paper, and to 1cm
% if using A4 paper. (This probably isn't strictly necessary.)
% If using another size paper, use default 1cm margins.
\ifthenelse{\lengthtest { \paperwidth = 11in}}
    { \geometry{top=.5in,left=.5in,right=.5in,bottom=.5in} }
    {\ifthenelse{ \lengthtest{ \paperwidth = 297mm}}
        {\geometry{top=1cm,left=1cm,right=1cm,bottom=1cm} }
        {\geometry{top=1cm,left=1cm,right=1cm,bottom=1cm} }
    }

% Turn off header and footer
\pagestyle{empty}

% Redefine section commands to use less space
\makeatletter
\renewcommand{\section}{\@startsection{section}{1}{0mm}%
                                {-1ex plus -.5ex minus -.2ex}%
                                {0.5ex plus .2ex}%x
                                {\normalfont\large\bfseries}}
\renewcommand{\subsection}{\@startsection{subsection}{2}{0mm}%
                                {-1explus -.5ex minus -.2ex}%
                                {0.5ex plus .2ex}%
                                {\normalfont\normalsize\bfseries}}
\renewcommand{\subsubsection}{\@startsection{subsubsection}{3}{0mm}%
                                {-1ex plus -.5ex minus -.2ex}%
                                {1ex plus .2ex}%
                                {\normalfont\small\bfseries}}
\makeatother

% Define BibTeX command
\def\BibTeX{{\rm B\kern-.05em{\sc i\kern-.025em b}\kern-.08em
    T\kern-.1667em\lower.7ex\hbox{E}\kern-.125emX}}

% Don't print section numbers
\setcounter{secnumdepth}{0}


\setlength{\parindent}{0pt}
\setlength{\parskip}{0pt plus 0.5ex}

%My Environments
\newtheorem{example}[section]{Example}
% -----------------------------------------------------------------------

\begin{document}
% CS 530 commands
	\newcommand{\conj}[1] {\overline{#1}}
	\newcommand{\cmod}[1] {|#1|}
	\newcommand{\norm}[1] {\|#1\|}
	\newcommand{\Complex}[0] {\mathbb{C}}
	% \fte{m}{n}{N}
	\newcommand{\fte}[3] {e^{2 \pi i \frac{#1*#2}{#3}}}
	\newcommand{\ftem}[3] {e^{2 \pi i \frac{#1 #2}{#3}}}
	\newcommand{\fteim}[3] {e^{-2 \pi i \frac{#1 #2}{#3}}}
	% \fteb{n}{N}
	\newcommand{\fteb}[2] {e^{2 \pi i \frac{#1}{#2}}}
	\newcommand{\ftei}[2] {e^{i \frac{#1 \pi}{#2}}}
	\newcommand{\ftein}[2] {e^{-i \frac{#1 \pi}{#2}}}
	\newcommand{\imagewidth}[0]{0.9\textwidth}
	\newcommand{\inp}[2]{\langle #1, #2 \rangle}
	\newcommand{\union}[0]{\cup}
% END CS 530 commands
\raggedright
\footnotesize
\begin{multicols}{3}


% multicol parameters
% These lengths are set only within the two main columns
%\setlength{\columnseprule}{0.25pt}
\setlength{\premulticols}{1pt}
\setlength{\postmulticols}{1pt}
\setlength{\multicolsep}{1pt}
\setlength{\columnsep}{2pt}

\begin{center}
     \Large{\underline{CS 530 Midterm 1 Cheatsheet}} \\
\end{center}

\section{Complex Analysis}
$z = a+ib = re^{i\theta} = r (cos(\theta) + i sin(\theta))$

$ \conj{z} = a - bi = re^{-i\theta}$

$\cmod{z} = \sqrt{a^2+b^2}$

Inner product $$\inp{z}{w} = \sum_{n=0}^{N-1} z(n)\conj{w(n)}$$

$\norm{\mathbf{v}} = \sqrt{\inp{\mathbf{v}}{\mathbf{v}}} $

Cauchy-Shwarz inequality
$\conj{\inp{u}{v}} \leq \norm{u} \norm{v}$

\section{Fourier Transform}
Fourier Basis for $\Complex^N $ is $\{F_0,F_1, \cdots, F_{N-1}\}$
$$
F_m(n) = \frac{1}{N} \ftem{m}{n}{N}
$$

Fourier basis is orthogonal, not orthonormal. Can be normalized the norm is $\norm{F_m} = \frac{1}{\sqrt{N}}$

$$
\hat{z}(m) = \sum_{n=0}^{N-1} z(n) \fteim{m}{n}{N}
$$

$$
z(n) = \frac{1}{N} \sum_{m=0}^{N-1} \hat{z}(m) \ftem{m}{n}{N}
$$

\subsection{2D}
$$
\hat{z}(m_1,m_2) = \frac{1}{\sqrt{N_1 N_2}} \sum_{n_1=0}^{N_1-1} \sum_{n_2=0}^{N_2-1} z(n_1,n_2) \fteim{m_1}{n_1}{N_1} \fteim{m_2}{n_2}{N_2}
$$

$$
z(n_1,n_2) = \frac{1}{\sqrt{N_1 N_2}} \sum_{m_1=0}^{N_1-1} \sum_{m_2=0}^{N_2-1} \hat{z}(m_1,m_2) \ftem{m_1}{n_1}{N_1} \ftem{m_2}{n_2}{N_2}
$$

\section{Wavelets}
\subsection{Wavelet tools}
$$(w*z)(m) = \sum_{n=0}^{N-1}w(n) z(m-n)$$

$(w*z)^{\wedge}(m) = \hat{w}(m) \hat{z}(m)$

$R_k z$ is called the translation of $z$ by $k$.

$ (R_kz)(n) = z(n-k) $

$(R_kz)^{\wedge}(m) = \hat{z}(m) \fteim{m}{k}{N}$

$ \inp{z}{R_kw} = (z * \tilde{w})(k) $

$ \inp{z}{R_k\tilde{w}} = (z * w)(k) $

$ \tilde{z}(n) = \conj{z(-n)} $

$(\tilde{z})^{\wedge}(m) = \conj{\tilde{z}(m)} $

$z^\ast(n) = (-1)^n z(n)$

For the below, $N = 2M$

$(z^\ast)^{\wedge}(n) = \hat{z}(n+M)$

\subsection{1st Stage Wavelet Basis}
$A(n)$ is the system matrix of $u$ and $v$

$$
A(n) =
\frac{1}{\sqrt{2}}
\begin{bmatrix}
\hat{u}(n) & \hat{v}(n) \\
\hat{u}(n + M) & \hat{v}(n + M)
\end{bmatrix}
$$

$$
B = \{R_{2k}v\}_{k=0}^{M-1} \union \{R_{2k}u\}_{k=0}^{M-1} 
$$
$$
= \{R_0v, R_2v, \dots, R_{N-2}v, R_0u, R_2 u, \dots, R_{N-2}u \}
$$

B (u and v) form a first stage wavelet basis if and only if the following properties are true:
\begin{enumerate}
\item $\cmod{\hat{u}(n)}^2 + \cmod{\hat{u}(n + M)}^2 = 2$ 
\item $\cmod{\hat{v}(n)}^2 + \cmod{\hat{v}(n + M)}^2 = 2$ 
\item $\hat{u}(n)\conj{\hat{v}(n)} + \hat{u}(n + M)\conj{\hat{v}(n + M)} $
\end{enumerate}

$ \forall n = 0,1,\cdots, M-1$

\subsection{1st Stage Shannon Basis}
$$
\hat{u} = \left\{ 
\begin{array}{cc}
\sqrt{2} & \text{if } n = 0, 1, \dots,\frac{N}{4}-1 \text{ or } n = \frac{3N}{4}, \frac{3N}{4}+1,\dots,N-1 \\
0 & \text{if } n = \frac{N}{4},\frac{N}{4}+1,\dots, \frac{3N}{4}-1
\end{array} \right.
$$

$$
\hat{v} = \left\{ 
\begin{array}{cc}
0 & \text{if } n = 0, 1, \dots,\frac{N}{4}-1 \text{ or } n = \frac{3N}{4}, \frac{3N}{4}+1,\dots,N-1 \\
\sqrt{2} & \text{if } n = \frac{N}{4},\frac{N}{4}+1,\dots, \frac{3N}{4}-1
\end{array} \right.
$$

\subsection{Real Wavelet Basis}

\subsubsection{1st Stage Real Shannon Basis}
$$
\hat{u} = \left\{ 
\begin{array}{cc}
\sqrt{2} & \text{if } n = 0, 1, \dots,\frac{N}{4}-1 \text{ or } n = \frac{3N}{4}+1,\dots,N-1 \\
i & \text{if } n = \frac{N}{4} \\
-i & \text{if } n = \frac{3N}{4} \\
0 & \text{if } n = \frac{N}{4}+1,\dots, \frac{3N}{4}-1
\end{array} \right.
$$

$$
\hat{v} = \left\{ 
\begin{array}{cc}
0 & \text{if } n = 0, 1, \dots,\frac{N}{4}-1 \text{ or } n = \frac{3N}{4}+1,\dots,N-1 \\
1 & \text{if } n = \frac{N}{4} \text{ or } n = \frac{3N}{4} \\
\sqrt{2} & \text{if } n = \frac{N}{4}+1,\dots, \frac{3N}{4}-1
\end{array} \right.
$$

\section{Misc}
$$
\sum_{k=a}^{b} r^{k} = \frac{r^a - r^{b+1}}{1-r}
$$

Unitary matrix
$ \iff \conj{U^T} U = I$

orthogonal $\iff \inp{u}{v} = 0 $

Parseval's formula 
$$
\inp{z}{w} = \frac{1}{N} \sum_{m=0}^{N-1} \hat{z}(m) \conj{\hat{w}(m)} = \frac{1}{N} \inp{\hat{z}}{\hat{w}}
$$

Plancherel's formula
$$
\norm{z}^2 = \frac{1}{N} \sum_{m=0}^{N-1} \cmod{\hat{z}(m)}^2 = \frac{1}{N} \norm{\hat{z}}^2
$$

\end{multicols}
\end{document}