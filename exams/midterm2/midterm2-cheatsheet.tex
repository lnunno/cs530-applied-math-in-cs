\documentclass[10pt,landscape]{article}
\usepackage{multicol}
\usepackage{calc}
\usepackage{ifthen}
\usepackage[landscape]{geometry}
\usepackage{amsmath,amsthm,amsfonts,amssymb}
\usepackage{color,graphicx,overpic}
\usepackage{hyperref}

% This file was found in the answer at this page: http://tex.stackexchange.com/questions/8827/preparing-cheat-sheets

\pdfinfo{
  /Title (example.pdf)
  /Creator (TeX)
  /Producer (pdfTeX 1.40.0)
  /Author (Lucas)
  /Subject (CS530 Midterm 2 Cheatsheet)
  /Keywords (pdflatex, latex,pdftex,tex)}

% This sets page margins to .5 inch if using letter paper, and to 1cm
% if using A4 paper. (This probably isn't strictly necessary.)
% If using another size paper, use default 1cm margins.
\ifthenelse{\lengthtest { \paperwidth = 11in}}
    { \geometry{top=.5in,left=.5in,right=.5in,bottom=.5in} }
    {\ifthenelse{ \lengthtest{ \paperwidth = 297mm}}
        {\geometry{top=1cm,left=1cm,right=1cm,bottom=1cm} }
        {\geometry{top=1cm,left=1cm,right=1cm,bottom=1cm} }
    }

% Turn off header and footer
\pagestyle{empty}

% Redefine section commands to use less space
\makeatletter
\renewcommand{\section}{\@startsection{section}{1}{0mm}%
                                {-1ex plus -.5ex minus -.2ex}%
                                {0.5ex plus .2ex}%x
                                {\normalfont\large\bfseries}}
\renewcommand{\subsection}{\@startsection{subsection}{2}{0mm}%
                                {-1explus -.5ex minus -.2ex}%
                                {0.5ex plus .2ex}%
                                {\normalfont\normalsize\bfseries}}
\renewcommand{\subsubsection}{\@startsection{subsubsection}{3}{0mm}%
                                {-1ex plus -.5ex minus -.2ex}%
                                {1ex plus .2ex}%
                                {\normalfont\small\bfseries}}
\makeatother

% Define BibTeX command
\def\BibTeX{{\rm B\kern-.05em{\sc i\kern-.025em b}\kern-.08em
    T\kern-.1667em\lower.7ex\hbox{E}\kern-.125emX}}

% Don't print section numbers
\setcounter{secnumdepth}{0}


\setlength{\parindent}{0pt}
\setlength{\parskip}{0pt plus 0.5ex}

%My Environments
\newtheorem{example}[section]{Example}
% -----------------------------------------------------------------------

\begin{document}
% CS 530 commands
	\newcommand{\conj}[1] {\overline{#1}}
	\newcommand{\cmod}[1] {|#1|}
	\newcommand{\norm}[1] {\|#1\|}
	\newcommand{\Complex}[0] {\mathbb{C}}
	\newcommand{\Real}[0] {\mathbb{R}}
	% \fte{m}{n}{N}
	\newcommand{\fte}[3] {e^{2 \pi i \frac{#1*#2}{#3}}}
	\newcommand{\ftem}[3] {e^{2 \pi i \frac{#1 #2}{#3}}}
	\newcommand{\fteim}[3] {e^{-2 \pi i \frac{#1 #2}{#3}}}
	% \fteb{n}{N}
	\newcommand{\fteb}[2] {e^{2 \pi i \frac{#1}{#2}}}
	\newcommand{\ftei}[2] {e^{i \frac{#1 \pi}{#2}}}
	\newcommand{\ftein}[2] {e^{-i \frac{#1 \pi}{#2}}}
	\newcommand{\imagewidth}[0]{0.9\textwidth}
	\newcommand{\inp}[2]{\langle #1, #2 \rangle}
	\newcommand{\union}[0]{\cup}
	\newcommand{\gradient}[0]{\nabla}
% END CS 530 commands
\raggedright
\footnotesize
\begin{multicols}{3}


% multicol parameters
% These lengths are set only within the two main columns
%\setlength{\columnseprule}{0.25pt}
\setlength{\premulticols}{1pt}
\setlength{\postmulticols}{1pt}
\setlength{\multicolsep}{1pt}
\setlength{\columnsep}{2pt}

\begin{center}
     \Large{\underline{CS 530 Midterm 2 Cheatsheet}} \\
\end{center}

\section{Wavelets}
\subsection{Wavelet tools}
$$(w*z)(m) = \sum_{n=0}^{N-1}w(m-n) z(n)$$

The convolution matrix is

$$
C =
\{R_0w,R_1w,\dots,R_{N-1}w\}
$$

For example, with $N = 4$

$$
C = 
\{R_0w,R_1w,R_2w,R_3w\}
$$

The convolution operation is simply

$$
(w*z)(m) = C \cdot z
$$

$(w*z)^{\wedge}(m) = \hat{w}(m) \hat{z}(m)$

$R_k z$ is called the translation of $z$ by $k$.

$ (R_kz)(n) = z(n-k) $

$(R_kz)^{\wedge}(m) = \hat{z}(m) \fteim{m}{k}{N}$

$ \inp{z}{R_kw} = (z * \tilde{w})(k) $

$ \inp{z}{R_k\tilde{w}} = (z * w)(k) $

$ \tilde{z}(n) = \conj{z(-n)} $

$(\tilde{z})^{\wedge}(m) = \conj{\tilde{z}(m)} $

$(\tilde{u}*\tilde{v})(n) = (u*v)\tilde{}(n)$

$z^\ast(n) = (-1)^n z(n)$

For the below, $N = 2M$

$(z^\ast)^{\wedge}(n) = \hat{z}(n+M)$

Unitary matrix
$ \iff \conj{U^T} U = I$

Parseval's formula 
$$
\inp{z}{w} = \frac{1}{N} \sum_{m=0}^{N-1} \hat{z}(m) \conj{\hat{w}(m)} = \frac{1}{N} \inp{\hat{z}}{\hat{w}}
$$

Plancherel's formula
$$
\norm{z}^2 = \frac{1}{N} \sum_{m=0}^{N-1} \cmod{\hat{z}(m)}^2 = \frac{1}{N} \norm{\hat{z}}^2
$$

\subsection{Up/Down Sampling}
Downsampling
$$
z = 
\begin{bmatrix}
z(0) \\
z(1) \\
z(2) \\
z(3) \\
z(4) \\
z(5) \\
z(6) \\
z(7)
\end{bmatrix}
\xrightarrow{\downarrow2}
D_z
=
\begin{bmatrix}
z(0) \\
z(2) \\
z(4) \\
z(6) 
\end{bmatrix}
$$


Upsampling

$$
z = 
\begin{bmatrix}
z(0) \\
z(1) \\
z(2) \\
z(3) 
\end{bmatrix}
\xrightarrow{\uparrow2}
U_z
=
\begin{bmatrix}
z(0) \\
0    \\
z(1) \\
0    \\
z(2) \\
0    \\
z(3) \\
0
\end{bmatrix}
$$

\subsubsection{Properties}
\begin{enumerate}
\item $D(z) * w = D(z*U(w))$
\item $U(x)*U(y) = U(x*y)$
\end{enumerate}

Suppose $N$ is divisible by $2^l,x,y,w\in l^2(Z_{N/2^l})$ and $z \in l^2(Z_N)$ Then
\begin{enumerate}
\item $D^l(z) * w = D^l(z*U^l(w))$
\item $U^l(x)*U^l(y) = U^l(x*y)$
\end{enumerate}

\subsection{1st Stage Wavelet Basis}
$A(n)$ is the system matrix of $u$ and $v$

$$
A(n) =
\frac{1}{\sqrt{2}}
\begin{bmatrix}
\hat{u}(n) & \hat{v}(n) \\
\hat{u}(n + M) & \hat{v}(n + M)
\end{bmatrix}
$$

$$
B = \{R_{2k}v\}_{k=0}^{M-1} \union \{R_{2k}u\}_{k=0}^{M-1} 
$$
$$
= \{R_0v, R_2v, \dots, R_{N-2}v, R_0u, R_2 u, \dots, R_{N-2}u \}
$$

B (u and v) form a first stage wavelet basis if and only if the following properties are true:
\begin{enumerate}
\item $\cmod{\hat{u}(n)}^2 + \cmod{\hat{u}(n + M)}^2 = 2$ 
\item $\cmod{\hat{v}(n)}^2 + \cmod{\hat{v}(n + M)}^2 = 2$ 
\item $\hat{u}(n)\conj{\hat{v}(n)} + \hat{u}(n + M)\conj{\hat{v}(n + M)} = 0$
\end{enumerate}

$ \forall n = 0,1,\cdots, M-1$

\subsection{1st Stage Shannon Basis}
$$
\hat{u} = \left\{ 
\begin{array}{cc}
\sqrt{2} & \text{if } n = 0, 1, \dots,\frac{N}{4}-1 \text{ or } n = \frac{3N}{4}, \frac{3N}{4}+1,\dots,N-1 \\
0 & \text{if } n = \frac{N}{4},\frac{N}{4}+1,\dots, \frac{3N}{4}-1
\end{array} \right.
$$

$$
\hat{v} = \left\{ 
\begin{array}{cc}
0 & \text{if } n = 0, 1, \dots,\frac{N}{4}-1 \text{ or } n = \frac{3N}{4}, \frac{3N}{4}+1,\dots,N-1 \\
\sqrt{2} & \text{if } n = \frac{N}{4},\frac{N}{4}+1,\dots, \frac{3N}{4}-1
\end{array} \right.
$$

\subsection{Real Wavelet Basis}

\subsubsection{1st Stage Real Shannon Basis}
$$
\hat{u} = \left\{ 
\begin{array}{cc}
\sqrt{2} & \text{if } n = 0, 1, \dots,\frac{N}{4}-1 \text{ or } n = \frac{3N}{4}+1,\dots,N-1 \\
i & \text{if } n = \frac{N}{4} \\
-i & \text{if } n = \frac{3N}{4} \\
0 & \text{if } n = \frac{N}{4}+1,\dots, \frac{3N}{4}-1
\end{array} \right.
$$

$$
\hat{v} = \left\{ 
\begin{array}{cc}
0 & \text{if } n = 0, 1, \dots,\frac{N}{4}-1 \text{ or } n = \frac{3N}{4}+1,\dots,N-1 \\
1 & \text{if } n = \frac{N}{4} \text{ or } n = \frac{3N}{4} \\
\sqrt{2} & \text{if } n = \frac{N}{4}+1,\dots, \frac{3N}{4}-1
\end{array} \right.
$$

\subsection{Wavelet Transform}

Wavelet transform
$$
z =
\left\{
\begin{array}{c}
* \tilde{u} \to z * \tilde{u} \xrightarrow{\downarrow2} \\
* \tilde{v} \to z * \tilde{v} \xrightarrow{\downarrow2}
\end{array}
\right.
=
\begin{bmatrix}
D(z * \tilde{u}) \\
D(z * \tilde{v}) \\
\end{bmatrix}
$$

Inverse wavelet transform

$$
\begin{bmatrix}
D(z * \tilde{u}) \\
D(z * \tilde{v}) \\
\end{bmatrix}
=
\begin{array}{c}
\xrightarrow{\uparrow2} \frac{(z*\tilde{u})+(z*\tilde{u})^{*}}{2}*u \\
\xrightarrow{\uparrow2} \frac{(z*\tilde{v})+(z*\tilde{v})^{*}}{2}*v
\end{array}
+
=
z
$$

\subsection{Iterative Wavelet Construction}
$$
z =
\left\{
\begin{array}{l}
* \tilde{v_1} \to \downarrow2 \xrightarrow{x_1} \\
* \tilde{u_1} \to \downarrow2 \xrightarrow{y_1} 
\left\{
\begin{array}{c}
* \tilde{v_2} \to \downarrow2 \xrightarrow{x_2} \\
* \tilde{u_2} \to \downarrow2 \xrightarrow{y_2}
\end{array}
\right.
\end{array}
\right.
$$

$ x_1 = D(z*\tilde{v_1}) $
 
$ y_1 = D(z*\tilde{u_1}) $

$ x_2 = D(y_1 * \tilde{v_2}) = D(D(z*\tilde{u_1})*\tilde{v_2}) = D^2(z*(u_1*Uv_2)\tilde{})$

$ y_2 = D(y_1 * \tilde{u_2}) = D(D(z*\tilde{u_1})*\tilde{u_2}) = D^2(z*(u_1*Uu_2)\tilde{})$

$
\text{Let } f_1 = v_1, g_1 = u_1
$

$
f_2 = g_1 * Uv_1, g_2 = g_1 * Uu_2
$

Then
$x_1 = D(z*\tilde{f_1})$

$y_1 = D(z*\tilde{g_1})$

$x_2 = D^2(z*\tilde{f_2})$

$y_2 = D^2(z*\tilde{g_2})$

$$
f_l = g_{l-1} * U^{l-1}(v_l)
$$

$$
f_\ell = u_1 * U(u_2) * U^2(u_3) * \cdots U^{\ell-2}(u_{\ell-1})*U^{\ell-1}(v_\ell)
$$

$$
g_l = g_{l-1}*U^{l-1}(u_l)
$$

$$
g_\ell = u_1 * U(u_2) * U^2(u_3) * \cdots U^{\ell-2}(u_{\ell-1})*U^{\ell-1}(u_\ell)
$$

$$
x_\ell = D^\ell(z*\tilde{f_\ell})
$$

$$
y_\ell = D^\ell(z*\tilde{g_\ell})
$$

The $\ell$th system matrix is

$$
A_\ell(n) =
\frac{1}{\sqrt{2}}
\begin{bmatrix}
\hat{u_\ell}(n) & \hat{v_\ell}(n) \\
\hat{u_\ell}(n + \frac{N}{2^\ell}) & \hat{v_\ell}(n + \frac{N}{2^\ell})
\end{bmatrix}
$$

\subsubsection{Haar construction}

$$
u_1 =
\begin{bmatrix}
\frac{1}{\sqrt{2}} \\
\frac{1}{\sqrt{2}} \\
0 \\
\vdots \\
0 
\end{bmatrix}
,
v_1 =
\begin{bmatrix}
\frac{1}{\sqrt{2}} \\
-\frac{1}{\sqrt{2}} \\
0 \\
\vdots \\
0 
\end{bmatrix}
$$

$$
u_\ell(0) = \frac{1}{\sqrt{2}}, u_\ell(1) = \frac{1}{\sqrt{2}}, \text{ and } u_\ell(n) = 0 \text{ for } 2 \leq n \leq (\frac{N}{2^{\ell-1}})-1
$$

$$
v_\ell(0) = \frac{1}{\sqrt{2}}, v_\ell(1) = \frac{1}{\sqrt{2}}, \text{ and } v_\ell(n) = 0 \text{ for } 2 \leq n \leq (\frac{N}{2^{\ell-1}})-1
$$

$$
f_{\ell}(n) =
\left\{
\begin{array}{ll}
2^{-\ell/2}, & n = 0,1,\dots,2^{\ell-1}-1 \\
-2^{-\ell/2}, & n = 2^{\ell-1},2^{\ell-1}+1,\dots,2^{\ell}-1 \\
0,           & n = 2^{\ell},2^{\ell}+1,\dots,N-1
\end{array}
\right.
$$

$$
g_{\ell}(n) =
\left\{
\begin{array}{ll}
2^{-\ell/2}, & n = 0,1,\dots,2^{\ell-1}-1 \\
0,           & n = 2^{\ell},2^{\ell}+1,\dots,N-1
\end{array}
\right.
$$

$
\psi_{-\ell,k}(n) =
\left\{
\begin{array}{ll}
2^{-\ell/2}, & n = 2^{\ell}k,2^{\ell}k+1,\dots,2^{\ell}k+2^{\ell-1}-1 \\
-2^{-\ell/2}, & n = 2^{\ell}k+2^{\ell-1},2^{\ell}k+2^{\ell-1}+1,\dots,2^{\ell}k+2^{\ell}-1 \\
0,           & n = 0,1,\dots,2^{\ell}k-1;2^{\ell}k+2^{\ell},\dots,N-1
\end{array}
\right.
$

$$
\varphi_{-\ell,k}(n) =
\left\{
\begin{array}{ll}
2^{-\ell/2}, & n = 2^{\ell}k,2^{\ell}k+1,\dots,2^{\ell}k+2^{\ell-1}-1 \\
0,           & n = 0,1,\dots,2^{\ell}k-1;2^{\ell}k+2^{\ell},\dots,N-1
\end{array}
\right.
$$

\subsubsection{Shannon construction}

$$
\hat{v}_{\ell}(n) =
\left\{
\begin{array}{ll}
\sqrt{2},    & \frac{N}{2^{\ell+1}} \leq n \leq \frac{3N}{2^{\ell+1}}-1 \\
0,           & 0 \leq n \leq \frac{N}{2^{\ell+1}}-1; \frac{3N}{2^{\ell+1}} \leq n \leq \frac{N}{2^\ell-1} - 1
\end{array}
\right.
$$

$$
\hat{u}_{\ell}(n) =
\left\{
\begin{array}{ll}
\sqrt{2},    & 0 \leq n \leq \frac{N}{2^{\ell+1}}-1; \frac{3N}{2^{\ell+1}} \leq n \leq \frac{N}{2^\ell-1} - 1 \\
0,           & \frac{N}{2^{\ell+1}} \leq n \leq \frac{3N}{2^{\ell+1}}-1
\end{array}
\right.
$$

\section{Optimization}
See Numerical Optimization Chapter 2.

In unconstrained optimization we minimize an objective function that depends on real variables, with no restrictions on these variables.

$$
\text{min}_x f(x)
$$

A point $x^*$ is a \textit{global minimizer} if $f(x^*) \leq f(x) \forall x$

A point $x^*$ is a \textit{local minimizer} if there is a neighborhood $N$ of $x^*$ such that $f(x^*) \leq f(x) \forall x \in N$

A point $x^*$ is a \textit{strict local minimizer} if there is a neighborhood $N$ of $x^*$ such that $f(x^*) < f(x) \forall x \in N \text{ with } x \neq x^*$

\subsection{Recognizing a Local Minimum}
Taylor's Theorem

$f : \Real^n \to \Real $ and $p \in \Real^n$

$$
f(x+p) = f(x) + \gradient f(x+tp)^Tp
$$

\subsubsection{First-Order Necessary Conditions}

If $x^*$ is a local minimizer and $f$ is continuously differentiable in an open neighborhood of $x^*$, then $\gradient f (x^*) = 0$

\subsubsection{Second-Order Necessary Conditions}

If $x^*$ is a local minimizer of $f$ and $\gradient^2f$ is continuous in an open neighborhood of $x^*$, then $\gradient f(x^*) = 0$ and $\gradient^2 f(x^*)$ is positive semidefinite.
\end{multicols}
\end{document}